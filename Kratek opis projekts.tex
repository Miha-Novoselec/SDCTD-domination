\documentclass[12pt]{article}

\usepackage[utf8]{inputenc}
\usepackage[slovene]{babel}
\usepackage{amsmath}
\usepackage{amsfonts}
\usepackage{amssymb}
\usepackage{graphicx}
\usepackage{geometry}
\usepackage{fancyhdr}
\usepackage{dsfont}



\geometry{a4paper, margin=1in}
\pagestyle{fancy}
\fancyhf{}
\rhead{SDCTD DOMINATION - kratek opis projekta}
\lhead{Miha Novoselec}
\rfoot{Stran \thepage}

\newcommand{\EE}{\mathbb{E}}
\newcommand{\PP}{\mathbb{P}}
\newcommand{\FF}{\mathfrak{F}}

\newtheorem{theorem}{Izrek}
\newtheorem{lemma}{Lema}
\newtheorem{definition}{Definicija}

\newenvironment{proof}{\paragraph{Dokaz:}}{\hfill$\square$}

\title{SDCTD DOMINATION - kratek opis projekta}
\author{Miha Novoselec, Nejc Rudež, Dimitrija Kostov}
\date{\today}

\begin{document}
\maketitle

\section{Uvod}
Tema naše naloge je SDCTD domination ("simultaneously dominating and complement total dominating"). Projekt bomo delali v programu Sage. Posamezne dele projekta bomo sproti objavljali na Github-u. 

\section{Osnovne definicije}
Za razumevanje teme projekta navedemo nekaj osnovnih definicij. Ukvarjali se bomo z neusmerjenimi grafi $G = (V, E)$.
\begin{definition}{Dominacijska množica na grafu}
    Podmnožica $D$ vozlišč grafa $G = (V, E)$ je \underline{dominacijska množica na grafu}, če velja, da je vsako vozlišče iz $G$ vozlišče v $D$ ali pa je sosed od vozlišča v $D$.
\end{definition}

\begin{definition}{Dominacijsko število grafa}
    Dominacijsko število grafa je velikost najmanjše dominacijske množice na grafu. Označimo ga z $\gamma(G)$.
\end{definition}

\begin{definition}{Totalna dominacijska množica na grafu}
    Podmnožica $D$ vozlišč grafa $G = (V, E)$ je dominacijska množica na grafu, če velja, da je vsako vozlišče iz $G$ sosed od vozlišča v $D$.
\end{definition}

\begin{definition}{Totalno dominacijsko število grafa}
    Dominacijsko število grafa je velikost najmanjše totalno dominacijske množice na grafu. Označimo ga z $\gamma_t(G)$.
\end{definition}

\begin{definition}{SDCTD dominacijska množica}
    Množico, ki hkrati dominira graf $G$ in totalno dominira komplement grafa $G$ (označujemo jo kot SDCTD množica) je množica $D$, 
    ki hkrati je hkrati dominacijska množica na $G$ in totalna dominacijska množica na $\overline{G}$.
\end{definition}

\begin{definition}{SDCTD število grafa $G$}
    Minimalna kardinalnost SDCTD množice grafa $G$ označimo z $\overline{\gamma}(G)$ in jo imenujemo SDCTD število grafa $G$. 
\end{definition}

\section{Načrt dela}
Najprej bomo implementirali osnovni CLP, ki sprejme neusmerjen graf $G = (V,E)$ in vrne najmanjšo SDCTD množico $D$ na grafu $G$. 

\begin{align*}
min \sum_{v \in V(G)} x_v &\\
s.t. \sum_{w \in N_{G}(v)} x_w + x_v &\geq 1 \quad \text{ za vsak } v \in V(G) \\
\sum_{w \in N_{\overline{G}}(v)} x_w &\geq 1 \quad \text{ za vsak } v \in V(G) \\
x_v &\in {0,1} \text{ za vsak } v \in V(G)
\end{align*}

\noindent CLP vrne $D = \{v \in V(G): x_v = 1\}$. 

\vspace{12pt}
\noindent V nalogi bomo nato predstavili rezultate, ki se bodo nanašali na lastnosti SDCTD število različnih grafov. V grobem se bomo ukvarjali z naslednjimi probelmi:
\begin{enumerate}
    \item Želeli bomo ugotoviti, kakšni grafi reda $n$ dosežejo največje (možno) in kateri najmanjše (možno) dominacijsko število.
    \item How does this invariant behave with regard to the Cartesian product and other graph pro-
    ducts? Can you generate a Vizing-type conjecture?
    \item Kaj se dogaja z $\overline{\gamma}(G)$, ko dodamo dodaten pogoj minimalne/maksimalne stopnje vozlišč grafa G.
    \item Izračunali bomo $\overline{\gamma}(G)$ za grafe premera 2.
    \item Ali lahko omejimo $\overline{\gamma}(G) + \overline{\gamma}(\overline{G})$ v odvisnosti od $n$.
\end{enumerate} 

Pri raziskovanju bomo za majhne grafe uporabljali sistematično iskanje, za večje pa stohastično iskanje.

\end{document}