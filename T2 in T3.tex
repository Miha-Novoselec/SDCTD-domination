\documentclass[a4paper, 12pt]{article}
\usepackage[margin=1in]{geometry} % Adjust margins here
\usepackage{amsmath}
\usepackage{amssymb}

\begin{document}

\section*{Točka 2: Kako se $\overline\gamma(G)$ obnaša glede na kartezični produkt in druge tipe produktov grafov? Ali lahko oblikujemo Vizingovo domnevo? }

Pri analizi lastnosti SDCTD števila glede na grafovne produkte se bomo osredotočili predvsem na kartezični produkt grafov. \\
Spomnimo se definicije kartezičnega produkta: 

\textbf{Kartezični produkt grafov}  \( G \) in \( H \) je graf \( G \square H \), kjer je množica vozlišč \( V(G \square H) = V(G) \times V(H) \), in kjer sta vozlišči \( (u_1, v_1) \) in \( (u_2, v_2) \) povezani, če velja eno izmed naslednjih dveh pravil:

\begin{itemize}
    \item \( u_1 = u_2 \) in \( v_1 \) je sosed od \( v_2 \) v grafu \( H \),
    \item \( v_1 = v_2 \) in \( u_1 \) je sosed od \( u_2 \) v grafu \( G \).
\end{itemize}

\subsection*{Kaj nas zanima?}

Želimo raziskati obnašanje \( \overline{\gamma}(G \square H) \) glede na \( \overline{\gamma}(G) \) in \( \overline{\gamma}(H) \): Želimo razumeti, kako SDCTD kartezičnega produkta dveh grafov \( G \) in \( H \) (tj. število \( \overline{\gamma}(G \square H) \)) odvisna od SDCTD številka posameznih grafov \( G \) in \( H \) (tj. \( \overline{\gamma}(G) \) in \( \overline{\gamma}(H) \)). \\

IDEJA: Narediti CLP, ki ustvari dva poljubna grafa in izračuna njun kartezični produkt. Nato za te tri grafe izračuna števil SDCTD. To izvedemo za več različnih grafov. Primerjamo vrednosti in skušamo ugotoviti vzorec ali zakonitost. Iz zbranih podatkov bomo prišli do Vizingove domneve. \\


Iz zbranih lahko predlagamo hipotezo o razmerju med \( \overline{\gamma}(G \square H) \), \( \overline{\gamma}(G) \) in \( \overline{\gamma}(H) \) - formiramo Vizingovo domnevo:

\textbf{\noindet Vizingova domneva:} Naj bosta \( G \) in \( H \) poljubna grafa. Potem velja:
\[
\overline{\gamma}(G \square H) \geq \overline{\gamma}(G) \cdot \overline{\gamma}(H)
\]
Graf \( G \) zadošča Vizingovi domnevi v primeru, ko zgornja neenakost velja za poljuben graf \( H \).

Vizingova domneva pravi, da je dominacijsko število kartezičnega produkta dveh grafov vsaj tako veliko kot produkt dominacijskih števil teh dveh grafov.







\newpage
\section*{Točka 3 –  Kaj se zgodi z $\overline\gamma(G)$, ko grafu, ki ga ustvarimo, dodamo dodatni pogoj minimalne/maksimalne stopnje vozlišč grafa \( G \)?}

Spomnimo se nekaj osnovnih pojmov: \\
\textbf{Najmanjša stopnja \( \delta(G) \)} grafa \( G \) je definirana kot najmanjša stopnja (število sosedov) katerega koli vozlišča v grafu. To pomeni, da \( \delta(G) \) daje informacijo o najmanjši povezanosti vozlišč v grafu. 

\textbf{ Največja stopnja \( \Delta(G) \)} grafa \( G \) je definirana kot največja stopnja katerega koli vozlišča v grafu. Višja največja stopnja \( \Delta(G) \) pomeni, da je eno vozlišče zelo povezano z drugimi (ima veliko sosedov). \\

V tem razdelku pogledamo, kako \( \overline{\gamma}(G) \) reagira na spremembe v stopnjah vozlišč grafa. \\
Ideja: Naredimo CLP, ki ustvari poljuben graf z n vozlišči. Nato v vsakem koraku spremninjamo število povezav za +1. Za vsak graf v vsakem korak izračunamo \(\overline{\gamma}(G) \) in analiziramo odvisnost od \( \delta(G) \) in \( \Delta(G)\). To lahko preverimo na manjšem grafu, če potrdimo hipotezo lahko pokažemo še na večjem grafu

\\

 Možna vprašanja:
\begin{itemize}
    \item Ali najmanjša stopnja grafa vpliva na možnost dominacije? 
    \item Ali se lahko zgodi, da postavimo tako največjo stopnjo grafa, da bi zmanjšali SDCTD številko?
\end{itemize}

\\



\end{document}
